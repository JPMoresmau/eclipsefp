\begin{hcarentry}[updated]{Haskell support for the Eclipse IDE}
\label{eclipsefp}
\report{Leif Frenzel}
\status{working, though alpha}
\entry{updated}% done, 30.10.2005
\makeheader

The Eclipse platform is an extremely extensible framework for IDEs,
developed by an Open Source Project. Our project extends it with tools
to support Haskell development.

The aim is to develop an IDE for Haskell that provides the set of features and
the user experience known from the Eclipse Java IDE (the flagship of the
Eclipse project), and integrates a broad range of compilers, interpreters,
debuggers, documentation generators and other Haskell development tools.
Long-term goals include a language model with support for language-aware IDE
features, like refactoring and structural search.

The current version is 0.9 (considered 'alpha'). It features a project model,
a configurable source code editor (with syntax coloring and code assist),
compiler support for GHC, interpreter support for GHCi and HUGS, documentation
generation with Haddock~\cref{haddock}, and launching from the IDE. In the
time between the last HC\&A report and now we have experimented in a number of
different directions, including Debugging support using the Eclipse Debug 
framework and the possibility to write part of the Haskell plugins in Haskell
itself. So far we are doing this with lots of handcoding using JNI and FFI. We
are trying to find a way to simplify and generalize this so that Eclipse plugins
can partly be written in Haskell. We have also started to add some design 
documents about this to the project homepage.

Every help is very welcome, be it in the form of code contributions, docs or
tutorials, or just any feedback if you use the IDE. If you want to
participate, please subscribe to the development mailing list (see below).

\FurtherReading
\begin{compactitem}
\item \url{http://eclipse.org}
\item \url{http://lists.sourceforge.net/lists/listinfo/eclipsefp-develop}
\item Project homepage: \url{http://eclipsefp.sf.net}
\end{compactitem}
\end{hcarentry}
